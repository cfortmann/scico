\documentclass{beamer}
\usepackage{pgfmath,pgffor}
\usepackage{tabularx}
\usepackage{arydshln}
\usepackage{placeins}
\usepackage{hyperref}
\usepackage{subcaption}
\usepackage[utf8]{inputenc}
\usepackage[T1]{fontenc}
\usepackage{amsmath}
\usepackage{amsthm}
\usepackage{amsfonts}
\usepackage{amssymb}
\usepackage{graphicx}
\usepackage{booktabs}
\usepackage{afterpage}
\usepackage[USenglish]{babel}
\usepackage{mdframed}
\usepackage{todonotes}
\usepackage{threeparttable}
\usepackage{framed}
\usepackage{xspace}	
\usepackage{fix-cm}

\usepackage[makeroom]{cancel}

\newcommand*{\tab}{&}
\newcommand*{\figca}[1]{\caption{#1}}

\usepackage{enumitem}
\setlist{leftmargin=3mm}
\usepackage{refcount}

\usepackage{cases}
\setlength{\parindent}{3em}

\usepackage{chngcntr}
\usepackage{multirow, multicol}
 \usepackage{float}

\counterwithin*{equation}{enumi}
\newcommand{\qr}{QR-Faktorisierung\xspace}
\newcommand{\luf}{LU-Faktorisierung\xspace}
\newcommand{\kui}{Kurven-Interpolation\xspace}
\newcommand{\kua}{Kurven-Approximation\xspace}
\newcommand{\chf}{Cholesky-Faktorisierung\xspace}
\newcommand{\schf}{Sparse \chf}
\usepackage{mathtools}
\DeclarePairedDelimiter\ceil{\lceil}{\rceil}
\usepackage{amsmath,amsfonts,amssymb,amsthm}
\usepackage{mathptmx}
\usepackage{calc}
\usepackage{wasysym}


\makeatletter
\def\pgfmathrandomitemwithoutreplacement#1#2{%
    \pgfmath@ifundefined{pgfmath@randomlist@#2}{\pgfmath@error{Unknown random list `#2'}}{%
        \edef\pgfmath@randomlistlength{\csname pgfmath@randomlist@#2\endcsname}%
        \ifnum\pgfmath@randomlistlength>0\relax%
            \pgfmathrandominteger{\pgfmath@randomtemp}{1}{\pgfmath@randomlistlength}%
            \def\pgfmath@marshal{\def#1}%
            \expandafter\expandafter\expandafter\pgfmath@marshal\expandafter\expandafter\expandafter{\csname pgfmath@randomlist@#2@\pgfmath@randomtemp\endcsname}%
            % Now prune.
            \c@pgfmath@counta=\pgfmath@randomtemp\relax%
            \c@pgfmath@countb=\c@pgfmath@counta%
            \advance\c@pgfmath@countb by1\relax%
            \pgfmathloop%
            \ifnum\c@pgfmath@counta=\pgfmath@randomlistlength\relax%
            \else%
                \def\pgfmath@marshal{\expandafter\global\expandafter\let\csname pgfmath@randomlist@#2@\the\c@pgfmath@counta\endcsname=}%
                \expandafter\pgfmath@marshal\csname pgfmath@randomlist@#2@\the\c@pgfmath@countb\endcsname%
                \advance\c@pgfmath@counta by1\relax%
                \advance\c@pgfmath@countb by1\relax%
            \repeatpgfmathloop%
            \advance\c@pgfmath@counta by-1\relax%
            \expandafter\xdef\csname pgfmath@randomlist@#2\endcsname{\the\c@pgfmath@counta}%        
        \else%
            \pgfmath@error{Random list `#2' is empty}%
        \fi%        
    }}


\def\pgfmathrandomlistcopy#1#2{%
    \pgfmath@ifundefined{pgfmath@randomlist@#2}{\pgfmath@error{Unknown random list `#2'}}{%
        \edef\pgfmath@randomlistlength{\csname pgfmath@randomlist@#2\endcsname}%
        \pgfmathloop%
        \ifnum\pgfmathcounter>\pgfmath@randomlistlength\relax%
        \else%
            \def\pgfmath@marshal{\expandafter\global\expandafter\let\csname pgfmath@randomlist@#1@\pgfmathcounter\endcsname=}%
            \expandafter\pgfmath@marshal\csname pgfmath@randomlist@#2@\pgfmathcounter\endcsname%
        \repeatpgfmathloop%
        \expandafter\xdef\csname pgfmath@randomlist@#1\endcsname{\pgfmath@randomlistlength}%
    }%  
}
\makeatother


\begin{document}
\pgfmathsetseed{\number\pdfrandomseed} % seed for random generator

\pgfmathdeclarerandomlist{WordsMaster}{%
 {\textbf{Wie werden $A$ und $b$ aufgestellt?\newline} \textcolor<1>{white}{$b$ beinhaltet die Werte $y_i$ der gegebenen Punkte. In $A$ werden die Basisfunktionen gespeichert, also zB. Monome (wie Übung) werden in der ersten Zeile die $x_1^{0...n}$ und in den Spalten $x_{1..n}$ gespeichert. Somit steht in der $i$-ten Zeile und $j$-ten Spalte der Eintrag $x_{i+1}^j$.} }%
 {\textbf{Was ist eine Basisfunktion?\newline} \textcolor<1>{white}{Eine Basisfunktion beschreibt eine Kurve mit einer endliche Anzahl von Parametern. Die von uns verwendete Basisfunktion ist die Monom-Basis $x^i$.} }%    
 {\textbf{Wie funktioniert \luf und wie löst man das Gleichungssystem?\newline} \textcolor<1>{white}{Faktorisierung von $A$ in untere ($L$) und obere Dreiecksmatrix ($U$) mittels Gau{\ss}.\[
	      \begin{pmatrix}
		* & * & * & *\\
		* & * & * & *\\
		* & * & * & *\\
		* & * & * & *
	      \end{pmatrix}
	      \rightarrow
	      \begin{pmatrix}
		* & * & * & *\\
		0 & * & * & *\\
		0 & * & * & *\\
		0 & * & * & *
	      \end{pmatrix}
	      \rightarrow
              \dots\]
	      Zum Lösen dann (1) $Ly = b$ und (2) $Ux = y$ lösen.\\
	      Vorwärtssubstitution für (1):\\
              Erstes Element: $y_1 = \frac{b_1}{l_{11}}$ \\
	      $\displaystyle y_i = \frac{1}{l_{ii}} (b_i - \sum_{j=1}^{i-1} l_{ij}y_{j} )$ für $i = 2, \dots, m$.\\
	      Rückwärts für (2): \\
	      $ x_1 = \frac{y_m}{u_{mm}}$ \\
	      $\displaystyle x_i = \frac{1}{u_{ii}} (y_i - \sum_{j=i+1}^{m} u_{ij}x_j)$ für $i = m-1, \dots, 1$. } }%
    {\textbf{ Wie kann man \luf stabiler machen?\newline} \textcolor<1>{white}{Pivotisierung der Matrix, mittels Permutationsmatrix. Vertausche den maximalen Eintrag $a_{ik}^k$ der Spalte mit dem Element auf der Diagonalen $a_{kk}^k$.} }    
    {\textbf{Wann geht der \luf kaputt?\newline} \textcolor<1>{white}{Die \luf scheitert sobald der Algorithmus versucht eine Division durch 0 durchzuführen. Bsp. Matrix: $\begin{pmatrix}
    		0&1\\
    		1&1
    		\end{pmatrix}$}}
    {\textbf{Was ist der Unterschied zwischen Approximation und Interpolation?\newline} \textcolor<1>{white}{Polynomgrad bzw. Dimensionen der Matrix $A^{m \times n}$.\\
    \begin{itemize}
    	\item  $m = n$: $A$ quadratisch, voller Rang, existiert Inverse, dann eindeutige Lösung. $\Rightarrow$ Interpolation
    	\item  $m > n$: System überbestimmt, nicht exakt lösbar. $\Rightarrow$ Approximation.  
    \end{itemize}} }  
	{\textbf{ Wie kann ein überbestimmtes Gleichungssystem approximiert werden?\newline} \textcolor<1>{white}{
	Normalengleichung mit anschließender \luf. oder \chf oder SVD oder ...		
	} } 	       
	{\textbf{ Warum sollte man manchmal ein Polynom kleineren Grades finden wollen?\newline} \textcolor<1>{white}{
	Durch ein Polynom kleineren Grades können verrauschte Daten besser approximiert werden.
	} }
	{\textbf{ Was ist der Grad eines Polynomes?\newline} \textcolor<1>{white}{
	Der Grad eines Polynomes entspricht dessen höchster Potenz.
	} }
{\textbf{ Was sind Normalengleichungen und wie sind sie definiert? Welche Dimension hat $A^TA$?\newline} \textcolor<1>{white}{
		Die Normalengleichung enthält die Lösung des überbestimmten (inkonsisten) Gleichungssystem $Ax=b$ und ist definiert als $A^TAx=A^Tb$. Dabei hat $A^TA$ die Dimension $n\times n$.
	} }
{\textbf{Wie leitet man die Normalengleichung her?\newline} \textcolor<1>{white}{
	Es gibt 3 Herleitungsarten, analytisch durch partielle Ableitungen oder Vektor-Ableitungen oder geometrisch. Die geometrische Herleitung geht vom folgenden aus:\\
	\includegraphics<2>[scale=0.3]{../GeoHer}
} }
{\textbf{ Was ist die 2-Norm?\newline} \textcolor<1>{white}{
	\[\|x\|_2=\sqrt{\sum_{i}x_i^2}\]
} }
{\textbf{ Wann ist eine Matrix symmetrisch positiv definit?\newline} \textcolor<1>{white}{
		Eine Matrix ist symmetrisch, wenn gilt, dass $A^T=A$ und positiv definit, wenn gilt, dass $\forall x\neq0: x^TAx>0$.
} }
{\textbf{ Welcher Löser muss verwendet werden, wenn $A$ nicht quadratisch ist?\newline} \textcolor<1>{white}{
		\chf, SVD oder \qr. Je nach Eigenschaft von $A$.
} }
{\textbf{ Was ist der Unterschied zwischen \chf und \qr?\newline} \textcolor<1>{white}{
	\qr ist besser konditioniert und stabiler ($\kappa(A)$) als die in der \chf verwendete Normalengleichung $A^TAx=A^Tb$ ($\kappa(A^TA)=\kappa(A)^2$).
} }
{\textbf{ Was ist die \chf? Wie funktioniert sie?\newline} \textcolor<1>{white}{
		\chf ist ein robuster und effizienter auf s.p.d. Matrizen basierender Löser, der mittels symmetrischer Gauß-Elimination funktioniert.\hfil
		\includegraphics<2>[scale=0.4]{../symgauss.png}\\
		Dabei wird die Matrix $A$ in die Faktoren $L,\;L^T$ zerlegt.Dabei wird die Matrix $A$ in die Faktoren $L$ und $L^T$ zerlegt.  $A=\begin{pmatrix}
		a_{11} & w^T \\
		w & K
		\end{pmatrix}
		=\begin{pmatrix}
		\sqrt{a_{11}} & 0 \\
		\frac{w}{\sqrt{a_{11}}} & I
		\end{pmatrix} 
		\begin{pmatrix}
		1 & 0 \\
		0 & \tilde{L}\tilde{L^T}
		\end{pmatrix}
		\begin{pmatrix}	
		\sqrt{a_{11}} & \frac{w^T}{\sqrt{a_{11}}} \\
		0 & I
		\end{pmatrix}   
		= LL^T$ 
} }
{\textbf{ Was ist fill-in und wie kommt es zustande?\newline} \textcolor<1>{white}{
	Fill-in bedeutet, dass ein Element $a_{ij}\in A$ 0 ist, jedoch nach der \chf $l_{ij}\neq0$ gilt. Das passiert, durch die gebildete Differenz $K-ww^T$. Fill-in tritt deshalb nur innerhalb des Bandes auf, alle Nullen außerhalb des Bandes bleiben erhalten.
} }
{\textbf{ Wie funktioniert die \qr?\newline} \textcolor<1>{white}{
		Erst orthonormale Basis $\lbrace q_1,..,q_n\rbrace$ von $range(A)$ konstruieren und anschließend die Vektoren aus $A$ mit der neuen Basis darstellen: $a_1=r_11q1,...,a_n=r_{1n}q_1+...+r_{nn}q_n$.\\
		Das ergibt in Matrix-Schreibweise: $A=QR$, sodass letztendlich folgt $Ax=b\Leftrightarrow QRx=QQ^Tb \Leftrightarrow Rx=Q^Tb$
} }
{\textbf{ Wie löst man das Least-Squares Problem mit \qr?\newline} \textcolor<1>{white}{
		\qr $A=QR$, berechne $b'=Q^Tb$ und löse $Rx=b'$ durch Rückwärtssubstitution
} }
{\textbf{ Was ist eine orthogonale Projektion?\newline} \textcolor<1>{white}{
		Entspricht der senkrechten Projektion eines Punktes $b$ auf das Bild einer Matrix $A$.
		Beginne mit Projektion von $b$ auf $\langle A\rangle$, mit $\|A\|=1$,
		$r=b-b'=(I-AA^T)b$, mit $b'=\left(\sum_{i=1}^{k}A_iA_i^T\right)b=\hat{A}\hat{A}^Tb$ und $\hat{A}\hat{A}^T$ als orthogonale Projektion.
} }
{\textbf{ Wann hat $A$ vollen Rang?\newline} \textcolor<1>{white}{
		Wenn die Zeilen- bzw. Spaltenvektoren linear unabhängig sind oder die Determinante von 0 verschieden ist.
} }
{\textbf{ Wie ist die Pseudoinverse definiert?\newline} \textcolor<1>{white}{
		$A^+=(A^TA)^{-1}A^T$, aber auch nur solange die Spalten linear unabhängig sind.
} }
{\textbf{ Was ist die Idee der Konditionierung?\newline} \textcolor<1>{white}{
		Das Problem als Funktion modellieren, welche bei Eingabe $x$ Ausgabe $f(x)$ liefert. Ein Problem ist gut konditioniert, wenn eine kleine Pertubation in $x$ auch nur eine kleine Pertubation in $f(x)$ bewirkt. Umgekehrt sollte klar sein. 
} }
}%

\newcommand*{\NumberOfQuizes}{20}%
\pgfmathrandomlistcopy{Words}{WordsMaster}

\foreach \QuizNumber in {1,...,\NumberOfQuizes} {%
    \pgfmathrandomitemwithoutreplacement{\RandomQuestion}{Words} 

\begin{frame}<1>
    \RandomQuestion
\end{frame}

\begin{frame}<2>
    \RandomQuestion
\end{frame}

}%

\end{document}
