\documentclass{beamer}
\usepackage{pgfmath,pgffor}
\usepackage{tabularx}
\usepackage{arydshln}
\usepackage{placeins}
\usepackage{hyperref}
\usepackage{subcaption}
\usepackage[utf8]{inputenc}
\usepackage[T1]{fontenc}
\usepackage{amsmath}
\usepackage{amsthm}
\usepackage{amsfonts}
\usepackage{amssymb}
\usepackage{graphicx}
\usepackage{booktabs}
\usepackage{arydshln}
\usepackage{afterpage}
\usepackage[USenglish]{babel}
\usepackage{mdframed}
\usepackage{todonotes}
\usepackage{threeparttable}
\usepackage{framed}
\usepackage{xspace}	
\usepackage{fix-cm}

\usepackage[makeroom]{cancel}

\newcommand*{\tab}{&}
\newcommand*{\figca}[1]{\caption{#1}}

\usepackage{enumitem}
\setlist{leftmargin=0mm}
\usepackage{refcount}

\usepackage{cases}
\setlength{\parindent}{0em}

\usepackage{chngcntr}
\usepackage{multirow, multicol}
 \usepackage{float}

\counterwithin*{equation}{enumi}
\newcommand{\qr}{QR-Faktorisierung\xspace}
\newcommand{\luf}{LU-Faktorisierung\xspace}
\newcommand{\kui}{Kurven-Interpolation\xspace}
\newcommand{\kua}{Kurven-Approximation\xspace}
\newcommand{\chf}{Cholesky-Faktorisierung\xspace}
\newcommand{\schf}{Sparse \chf}
\usepackage{mathtools}
\DeclarePairedDelimiter\ceil{\lceil}{\rceil}
\usepackage{amsmath,amsfonts,amssymb,amsthm}
\usepackage{mathptmx}
\usepackage{calc}
\usepackage{wasysym}


\makeatletter
\def\pgfmathrandomitemwithoutreplacement#1#2{%
    \pgfmath@ifundefined{pgfmath@randomlist@#2}{\pgfmath@error{Unknown random list `#2'}}{%
        \edef\pgfmath@randomlistlength{\csname pgfmath@randomlist@#2\endcsname}%
        \ifnum\pgfmath@randomlistlength>0\relax%
            \pgfmathrandominteger{\pgfmath@randomtemp}{1}{\pgfmath@randomlistlength}%
            \def\pgfmath@marshal{\def#1}%
            \expandafter\expandafter\expandafter\pgfmath@marshal\expandafter\expandafter\expandafter{\csname pgfmath@randomlist@#2@\pgfmath@randomtemp\endcsname}%
            % Now prune.
            \c@pgfmath@counta=\pgfmath@randomtemp\relax%
            \c@pgfmath@countb=\c@pgfmath@counta%
            \advance\c@pgfmath@countb by1\relax%
            \pgfmathloop%
            \ifnum\c@pgfmath@counta=\pgfmath@randomlistlength\relax%
            \else%
                \def\pgfmath@marshal{\expandafter\global\expandafter\let\csname pgfmath@randomlist@#2@\the\c@pgfmath@counta\endcsname=}%
                \expandafter\pgfmath@marshal\csname pgfmath@randomlist@#2@\the\c@pgfmath@countb\endcsname%
                \advance\c@pgfmath@counta by1\relax%
                \advance\c@pgfmath@countb by1\relax%
            \repeatpgfmathloop%
            \advance\c@pgfmath@counta by-1\relax%
            \expandafter\xdef\csname pgfmath@randomlist@#2\endcsname{\the\c@pgfmath@counta}%        
        \else%
            \pgfmath@error{Random list `#2' is empty}%
        \fi%        
    }}


\def\pgfmathrandomlistcopy#1#2{%
    \pgfmath@ifundefined{pgfmath@randomlist@#2}{\pgfmath@error{Unknown random list `#2'}}{%
        \edef\pgfmath@randomlistlength{\csname pgfmath@randomlist@#2\endcsname}%
        \pgfmathloop%
        \ifnum\pgfmathcounter>\pgfmath@randomlistlength\relax%
        \else%
            \def\pgfmath@marshal{\expandafter\global\expandafter\let\csname pgfmath@randomlist@#1@\pgfmathcounter\endcsname=}%
            \expandafter\pgfmath@marshal\csname pgfmath@randomlist@#2@\pgfmathcounter\endcsname%
        \repeatpgfmathloop%
        \expandafter\xdef\csname pgfmath@randomlist@#1\endcsname{\pgfmath@randomlistlength}%
    }%  
}
\makeatother


\begin{document}
\pgfmathsetseed{\number\pdfrandomseed} % seed for random generator

\pgfmathdeclarerandomlist{WordsMaster}{%
 {\textbf{Wie werden $A$ und $b$ aufgestellt?\newline} \textcolor<1>{white}{$b$ beinhaltet die Werte $y_i$ der gegebenen Punkte. In $A$ werden die Basisfunktionen gespeichert, also zB. Monome (wie Übung) werden in der ersten Zeile die $x_1^{0...n}$ und in den Spalten $x_{1..n}$ gespeichert. Somit steht in der $i$-ten Zeile und $j$-ten Spalte der Eintrag $x_{i+1}^j$.} }%
 {\textbf{Was ist eine Basisfunktion?\newline} \textcolor<1>{white}{Eine Basisfunktion beschreibt eine Kurve mit einer endliche Anzahl von Parametern. Die von uns verwendete Basisfunktion ist die Monom-Basis $x^i$.} }%    
 {\textbf{Wie funktioniert \luf und wie löst man das Gleichungssystem?\newline} \textcolor<1>{white}{Faktorisierung von $A$ in untere ($L$) und obere Dreiecksmatrix ($U$) mittels Gau{\ss}.\[
	      \begin{pmatrix}
		* & * & * & *\\
		* & * & * & *\\
		* & * & * & *\\
		* & * & * & *
	      \end{pmatrix}
	      \rightarrow
	      \begin{pmatrix}
		* & * & * & *\\
		0 & * & * & *\\
		0 & * & * & *\\
		0 & * & * & *
	      \end{pmatrix}
	      \rightarrow
              \dots\]
	      Zum Lösen dann (1) $Ly = b$ und (2) $Ux = y$ lösen.\\
	      Vorwärtssubstitution für (1):\\
              Erstes Element: $y_1 = \frac{b_1}{l_{11}}$ \\
	      $\displaystyle y_i = \frac{1}{l_{ii}} (b_i - \sum_{j=1}^{i-1} l_{ij}y_{j} )$ für $i = 2, \dots, m$.\\
	      Rückwärts für (2): \\
	      $ x_1 = \frac{y_m}{u_{mm}}$ \\
	      $\displaystyle x_i = \frac{1}{u_{ii}} (y_i - \sum_{j=i+1}^{m} u_{ij}x_j)$ für $i = m-1, \dots, 1$. } }%
    {\textbf{ Wie kann man \luf stabiler machen?\newline} \textcolor<1>{white}{Pivotisierung der Matrix, mittels Permutationsmatrix. Vertausche den maximalen Eintrag $a_{ik}^k$ der Spalte mit dem Element auf der Diagonalen $a_{kk}^k$.} }    
    {\textbf{Wann geht der \luf kaputt?\newline} \textcolor<1>{white}{Die \luf scheitert sobald der Algorithmus versucht eine Division durch 0 durchzuführen. Bsp. Matrix: $\begin{pmatrix}
    		0&1\\
    		1&1
    		\end{pmatrix}$}}
    {\textbf{Was ist der Unterschied zwischen Approximation und Interpolation?\newline} \textcolor<1>{white}{Polynomgrad bzw. Dimensionen der Matrix $A^{m \times n}$.\\
    \begin{itemize}
    	\item  $m = n$: $A$ quadratisch, voller Rang, existiert Inverse, dann eindeutige Lösung. $\Rightarrow$ Interpolation
    	\item  $m > n$: System überbestimmt, nicht exakt lösbar. $\Rightarrow$ Approximation.  
    \end{itemize}} }  
	{\textbf{ Wie kann ein überbestimmtes Gleichungssystem approximiert werden?\newline} \textcolor<1>{white}{
	Normalengleichung mit anschließender \luf. oder \chf oder SVD oder ...		
	} } 	       
	{\textbf{ Warum sollte man manchmal ein Polynom kleineren Grades finden wollen?\newline} \textcolor<1>{white}{
	Durch ein Polynom kleineren Grades können verrauschte Daten besser approximiert werden.
	} }
	{\textbf{ Was ist der Grad eines Polynomes?\newline} \textcolor<1>{white}{
	Der Grad eines Polynomes entspricht dessen höchster Potenz.
	} }
{\textbf{ Was sind Normalengleichungen und wie sind sie definiert? Welche Dimension hat $A^TA$?\newline} \textcolor<1>{white}{
		Die Normalengleichung enthält die Lösung des überbestimmten (inkonsisten) Gleichungssystem $Ax=b$ und ist definiert als $A^TAx=A^Tb$. Dabei hat $A^TA$ die Dimension $n\times n$.
	} }
{\textbf{Wie leitet man die Normalengleichung her?\newline} \textcolor<1>{white}{
	Es gibt 3 Herleitungsarten, analytisch durch partielle Ableitungen oder Vektor-Ableitungen oder geometrisch. Die geometrische Herleitung geht vom folgenden aus:\\
	\includegraphics<2>[scale=0.3]{../GeoHer}
} }
{\textbf{ Was ist die 2-Norm?\newline} \textcolor<1>{white}{
	\[\|x\|_2=\sqrt{\sum_{i}x_i^2}\]
} }
{\textbf{ Wann ist eine Matrix symmetrisch positiv definit?\newline} \textcolor<1>{white}{
		Eine Matrix ist symmetrisch, wenn gilt, dass $A^T=A$ und positiv definit, wenn gilt, dass $\forall x\neq0: x^TAx>0$.
} }
{\textbf{ Welcher Löser muss verwendet werden, wenn $A$ nicht quadratisch ist?\newline} \textcolor<1>{white}{
		\chf, SVD oder \qr. Je nach Eigenschaft von $A$.
} }
{\textbf{ Was ist der Unterschied zwischen \chf und \qr?\newline} \textcolor<1>{white}{
	\qr ist besser konditioniert und stabiler ($\kappa(A)$) als die in der \chf verwendete Normalengleichung $A^TAx=A^Tb$ ($\kappa(A^TA)=\kappa(A)^2$).
} }
{\textbf{ Was ist die \chf? Wie funktioniert sie?\newline} \textcolor<1>{white}{
		\chf ist ein robuster und effizienter auf s.p.d. Matrizen basierender Löser, der mittels symmetrischer Gauß-Elimination funktioniert.\hfil
		\includegraphics<2>[scale=0.4]{../symgauss.png}\\
		Dabei wird die Matrix $A$ in die Faktoren $L,\;L^T$ zerlegt.Dabei wird die Matrix $A$ in die Faktoren $L$ und $L^T$ zerlegt.  $A=\begin{pmatrix}
		a_{11} & w^T \\
		w & K
		\end{pmatrix}
		=\begin{pmatrix}
		\sqrt{a_{11}} & 0 \\
		\frac{w}{\sqrt{a_{11}}} & I
		\end{pmatrix} 
		\begin{pmatrix}
		1 & 0 \\
		0 & \tilde{L}\tilde{L^T}
		\end{pmatrix}
		\begin{pmatrix}	
		\sqrt{a_{11}} & \frac{w^T}{\sqrt{a_{11}}} \\
		0 & I
		\end{pmatrix}   
		= LL^T$ 
} }
{\textbf{ Was ist fill-in und wie kommt es zustande?\newline} \textcolor<1>{white}{
	Fill-in bedeutet, dass ein Element $a_{ij}\in A$ 0 ist, jedoch nach der \chf $l_{ij}\neq0$ gilt. Das passiert, durch die gebildete Differenz $K-ww^T$. Fill-in tritt deshalb nur innerhalb des Bandes auf, alle Nullen außerhalb des Bandes bleiben erhalten.
} }
{\textbf{ Wie funktioniert die \qr?\newline} \textcolor<1>{white}{
		Erst orthonormale Basis $\lbrace q_1,..,q_n\rbrace$ von $range(A)$ konstruieren und anschließend die Vektoren aus $A$ mit der neuen Basis darstellen: $a_1=r_11q1,...,a_n=r_{1n}q_1+...+r_{nn}q_n$.\\
		Das ergibt in Matrix-Schreibweise: $A=QR$, sodass letztendlich folgt $Ax=b\Leftrightarrow QRx=QQ^Tb \Leftrightarrow Rx=Q^Tb$
} }
{\textbf{ Wie löst man das Least-Squares Problem mit \qr?\newline} \textcolor<1>{white}{
		\qr $A=QR$, berechne $b'=Q^Tb$ und löse $Rx=b'$ durch Rückwärtssubstitution
} }
{\textbf{ Was ist eine orthogonale Projektion?\newline} \textcolor<1>{white}{
		Entspricht der senkrechten Projektion eines Punktes $b$ auf das Bild einer Matrix $A$.
		Beginne mit Projektion von $b$ auf $\langle A\rangle$, mit $\|A\|=1$,
		$r=b-b'=(I-AA^T)b$, mit $b'=\left(\sum_{i=1}^{k}A_iA_i^T\right)b=\hat{A}\hat{A}^Tb$ und $\hat{A}\hat{A}^T$ als orthogonale Projektion.
} }
{\textbf{ Wann hat $A$ vollen Rang?\newline} \textcolor<1>{white}{
		Wenn die Zeilen- bzw. Spaltenvektoren linear unabhängig sind oder die Determinante von 0 verschieden ist.
} }
{\textbf{ Wie ist die Pseudoinverse definiert?\newline} \textcolor<1>{white}{
		$A^+=(A^TA)^{-1}A^T$, aber auch nur solange die Spalten linear unabhängig sind.
} }
{\textbf{ Was ist die Idee der Konditionierung?\newline} \textcolor<1>{white}{
		Das Problem als Funktion modellieren, welche bei Eingabe $x$ Ausgabe $f(x)$ liefert. Ein Problem ist gut konditioniert, wenn eine kleine Pertubation in $x$ auch nur eine kleine Pertubation in $f(x)$ bewirkt. Umgekehrt sollte klar sein. 
} }
{\textbf{ Was bedeutet es, wenn ein Problem gut konditioniert ist?\newline} \textcolor<1>{white}{
		Ein Problem ist gut konditioniert, wenn eine kleine Pertubation in $x$ auch nur eine kleine Pertubation in $f(x)$ bewirkt. Umgekehrt sollte klar sein. Die Pertubartionen werden in realtiven Änderungsraten gemessen mit:$\frac{\|\delta x\|}{\|x\|}, \; \frac{\|f(x+\delta x)-f(x)\|}{\|f(x)\|}$.
} }
{\textbf{ Was sind finite Differenzen?\newline} \textcolor<1>{white}{
		\begin{align*}
			\text{Vorwärtsdiff.} &f'[i]&\approx\frac{f[i+1]-f[i]}{h}\\
			\text{Rückwärtsdiff.}&f'[i]&\approx\frac{f[i]-f[i-1]}{h}\\
			\text{Zentrale Diff.}&f'[i]&\approx\frac{f[i+1]-f[i-1]}{2h}
		\end{align*}
		$\mathop{\!\mathbin\bigtriangleup} u[i,j,t]=\frac{u[i-1,j,t]+u[i+1,j,t]+u[i,j-1,t]+u[i,j+1.t]-4u[i,j,t]}{h^2}$
} }
{\textbf{ Was sind parabolische, hyperbolische und elliptische PDEs?\newline} \textcolor<1>{white}{
		Parabolische PDEs beschreiben dynamische Gleichungen die gegen ein statisches Equilibrium konvergieren. Elliptische PDEs beschreiben statische Gleichgewichtszustände und hyperbolische modellieren dynamische Systeme ohne Gleichgewicht.
} }
{\textbf{ Was besagt Diskretisierung?\newline} \textcolor<1>{white}{
		Grundlegende Idee: Versuch eine Kurve mit einer endlichen Anzahl an Parametern zu beschreiben.
		Die Kurve wird mittels Basisfunktion $\varphi_j$ und Koeffizienten $f_j$ dargestellt: $\displaystyle f(x)=\sum_{j=1}^n f_j\varphi_j(x)$. 
} }
{\textbf{ Wie lässt sich die erste Ableitung mittels finiter Differenzen approximieren?\newline} \textcolor<1>{white}{
		Taylor-Entwicklung erster Ordnung $f(x)+hf'(x)+O(h^2)$ nach $f'(x)$ auflösen $\Rightarrow f'(x)\approx\frac{f(x+h)-f(x)}{h}$. Wird $f(x)$ diskretisiert mit $f[i]=f(ih)$ ergibt sich die Vorwärts- und Rückwärtsdifferenz, sowie die Zentrale Differenz.
} }
{\textbf{ Was ist der Unterschied zwischen $\mathop{\!\mathbin\bigtriangleup} u[i,j,t],\;-\mathop{\!\mathbin\bigtriangleup} u[i,j,t]$?\newline} \textcolor<1>{white}{
		$\mathop{\!\mathbin\bigtriangleup} u[i,j,t]$: Nachbarn addiert und dann das Zentrum 4-mal abziehen.\\
		$-\mathop{\!\mathbin\bigtriangleup} u[i,j,t]$: Zentrum mit 4 multiplizieren und dann die Nachbarn abziehen.
} }
{\textbf{Wie sieht die Heat-Equation aus?\newline} \textcolor<1>{white}{
		$\dot{u}=\mathop{\!\mathbin\bigtriangleup}u$
} }
{\textbf{Was sagt die Heat-Equation aus?\newline} \textcolor<1>{white}{
		Die Heat-Equation beschreibt den Fluss von Wärme in Richtung der Nachbarn.
} }
{\textbf{Was ist der implizite Euler? (HEAT)\newline} \textcolor<1>{white}{
		$u[i,j,t+1]=u[i,j,t]+\delta t \cdot\mathop{\!\mathbin\bigtriangleup}u[i,j,t+1]$\\
		\textbf{Vorteil:} stabil für jeden Zeitschritt, \textbf{Nachteil:} hoher Rechenaufwand
} }
{\textbf{Was ist der explizite Euler? (HEAT)\newline} \textcolor<1>{white}{
		Berechne $v[i,j]=t\mathop{\!\mathbin\bigtriangleup}u=\dot{u}$ und anschließend $u[i,j,t+1]=u[i,j,t]+\delta t\cdot \mathop{\!\mathbin\bigtriangleup}u[i,j,t]$\\
		\textbf{Vorteil:} schnell und einfach zu implementieren, \textbf{Nachteil:} instabil bei großen Zeitsprüngen, da $\delta t$ nicht mehr konvergiert, der Fehler stark zunimmt und somit keine Lösung mehr möglich ist.
} }
{\textbf{Was passiert bei zu hohen Schritten im expliziten Euler?\newline} \textcolor<1>{white}{
		Er konvergiert nicht mehr, da die Nachbarn einen zu hohen Einfluss auf die Lösung haben. Um ihn trotzdem zu berechnen, müsste man mehr Nachbarn in Betracht ziehen, da bei zu hohen Zeitschritten auch weit entfernte Nachbarn einen Einfluss auf den betrachteten Punkt haben.
} }
{\textbf{Was ist die CFL-Bedingung?\newline} \textcolor<1>{white}{
		$\delta t\leq\frac{h}{\|v\|}$, mit Informationskonstante $\|v\|$.
} }
{\textbf{Wann ist der Gleichgewichtszustand erreicht?\newline} \textcolor<1>{white}{
		Der implizite Euler konvergiert immer gegen den statischen Gleichgewichtszustand der Heat-Equation, der durch die Laplace-Equation $\mathop{\!\mathbin\bigtriangleup}u=0$ beschrieben wird. Während der explizite Euler nur bei ausreichend kleinen Zeitschritten konvergiert.
} }
{\textbf{Was passiert mit den Randpunkten? (HEAT)\newline} \textcolor<1>{white}{
		Die Randpunkte bleiben konstant.
} }
{\textbf{Wie stellt man das lineare Gleichungssystem für die Laplace-Gleichung auf?\newline} \textcolor<1>{white}{
	Jeder Punkt im Gitter ist eine Unbekannte, die berechnet werden muss. Wir haben also bei einem $(N$ x $N)$ Gitter $(N-2)$x$(N-2)$ Unbekannte (Randpunkte verändern sich nicht). Dies führt zu einem $(N-2)^2$ x $(N-2)^2$ Gleichungssystem $-Lu=b$.
	$u$ ist ein Spaltenvektor in dem die neuen Werte $u$ für das Gitter stehen. Die Reihenfolge, in der die Gitterpunkte in $u$ stehen, führt zu einer Index Abbildung: $u_k=u(i,j)$ mit $k=idx(i,j)=(i-1)+(j-2)\cdot(N-2)$.
	Für die Matrix L gilt: $l_{kk}=4, l_{kl}=-1/h^2$, wenn (\={i}, \={j}) Nachbarn von i, j und $l$=idx(\={i},\={j}) sonst $l_{kl}=0$.
} }
{\textbf{Warum stellt man das lineare Gleichungssystem für den auf?\newline} \textcolor<1>{white}{
 Wir wollen nur das Gleichgewicht $\mathop{\!\mathbin\bigtriangleup}u=0$ betrachten, welches nicht mit expliziter Integration lösbar ist. Der Grund ist, dass es sich dabei um eine statische PDE handelt, die nicht mehr von der Zeit abhängt.
} }
{\textbf{Wie löst man die Laplace-Gleichung?\newline} \textcolor<1>{white}{
		Die Laplace-Gleichung ist mittels \chf lösbar. Die Zeitkomplexität ist allerdings mit $\mathcal{O(n^3)}$ recht hoch, sodass unter anderem konjugierte Gradienten verwendet werden können. 
} }
{\textbf{Welche Eigenschaft hat die Laplace-Matrix $L$?\newline} \textcolor<1>{white}{
	Die $L$ ist sparse: Maximal sind 5 Einträge pro Zeile ungleich 0, $L$ ist symmetrisch und positiv/negativ definit.
} }
{\textbf{Wie funktionieren konjugierte Gradienten?}\newline \textcolor<1>{white}{
	beliebige Suchrichtung $p_n$ für die Minimierung von $f(x)$ zulassen: $x_{n+1}+\alpha_np_n$, wobei $\alpha_n$ ähnlich zum Gradientenabstieg bestimmt werde kann: $\alpha_n=\frac{r_n^Tr_n}{p_n^TAp_n}$. Wenn $p_n$ gleich dem negativen Gradienten $-\nabla f(x)=r_n$ entspricht, ist dies äquiv. zum Gradientenabstieg. 
} }
{\textbf{Wann werden konjugierte Gradienten angewendet?\newline} \textcolor<1>{white}{
	Konjugierte Gradienten werden angewendet, wenn die Matrix $A^(m\times m)$ sehr groß ist. (z.B.: $m>10000$).
} }
{\textbf{Wofür steht bei konjugierten Gradienten der Parameter $\alpha$?\newline} \textcolor<1>{white}{
	$\alpha$ steht für Schrittweite des Abstiegs. Beim Gradientenabstieg ist $\alpha=\frac{r_n^Tr_n}{r_n^TAr_n}$, bei konjugierten Gradienten ist $\alpha=\frac{r_n^Tr_n}{p_n^TAp_n}$.
} }
{\textbf{Welche Formen von Konvergenz gibt es beim Gradientenabstieg?\newline} \textcolor<1>{white}{
		\newline
		\begin{tabularx}{\textwidth}{X|X}
			Sphärische Konvergenz &Konvergenz in einem Schritt\\\hdashline
			Elliptische Konvergenz, \mbox{Startpunkt auf Hauptachse} &Konvergenz in einem Schritt\\\hdashline
			Elliptische Konvergenz, \mbox{beliebiger Startpunkt} &Langsame Konvergenz
		\end{tabularx}
} }
{\textbf{Wie ist der Gradientenabstieg definiert?\newline} \textcolor<1>{white}{
		Löse $Ax=b, \text{ mit } A \;s.p.d.$ formuliert als Minimierung $f(x)=\frac{1}{2}x^TAx-b^Tx \rightarrow \text{min}$.\\
		$f(x)$ hat ein Minimum, wenn gilt: $0=\nabla f(x)=Ax-b$.\\
		Minimiere $f(x)$ iterativ, indem bei jedem Schritt von $x_n$ ein Stückchen im Paraboloid heruntergelaufen wird. Dies entspricht dem negativen Gradienten $-\nabla f(x)$.\\
		Da $r_n=b-Ax_n=-\nabla f(x)$ folgt $r_n$ als Abstiegsrichtung der n-ten Iteration und $x_{n+1}=x_n+\alpha_nr_n$, mit $\alpha_n=\text{arg min}_\alpha f(x_{n+1})$.
} }
}%

\newcommand*{\NumberOfQuizes}{46}%
\pgfmathrandomlistcopy{Words}{WordsMaster}

\foreach \QuizNumber in {1,...,\NumberOfQuizes} {%
    \pgfmathrandomitemwithoutreplacement{\RandomQuestion}{Words} 

\begin{frame}<1>
    \RandomQuestion
\end{frame}

\begin{frame}<2>
	\raggedright
    \RandomQuestion
\end{frame}

}%

\end{document}
